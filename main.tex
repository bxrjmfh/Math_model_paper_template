\documentclass{my_paper}
\usepackage{ctex}
\usepackage[textwidth=444bp,vmargin=2.5cm]{geometry}%设置页边距
\usepackage{array} %主要是增加列样式选项
\usepackage[dvipsnames]{xcolor}%颜色宏包
\usepackage{graphicx}%图片宏包
\usepackage{amsmath}%公式宏包
\usepackage[T1]{fontenc}    
\usepackage{newtxtext, newtxmath}  %两种使用Times New Roman 字体的方法
\usepackage{subfigure}
\usepackage { gensymb }
% 打°符号\degree
\usepackage{listings}
% 代码

\begin{document}

\newpage
\begin{center}
\lunwenbiaoti

\vspace{2ex}
\zhaiyao
\end{center}

摘要

\begin{guanjianci}
 元胞自动机 \quad 边缘检测 \quad 形状匹配
\end{guanjianci}

%----------- 正文 ----------
%----------- 一、问题重述 ----------
\newpage
\section{一、问题重述}

\subsection{问题背景}

一些背景

\subsection{问题重述}
经过分析整理,我们需要解决以下问题:
\begin{enumerate}
    \item 考虑到
    \item 在自卸
    \item 在第二
\end{enumerate}
\section{二、问题分析}
\subsection{问题一的分析}

该问要求我们查阅参考资料,建立战机机动的量化模型。为此考虑根据已知的飞行参数确定对应的战斗机机动行为。一个战斗行为可能来带一系列参数的变化,因此结合多个参数的变化情况来判断动作类型,对于某些复杂动作,可能会有多段变化过程,因此需要结合相邻的两次变化情况进行判断。由于决定飞机飞行动作的参数种类有限,每一种参数也只有集中情况,考虑就飞行参数变化建立决策树,对战斗机的机动行为进行识别。

\subsection{问题二的分析}

分析题中附件所给数据,得出附件中信息有以下特点:数据冗余、数据缺失和个体多样问题。针对数据冗余问题,需要去除那些不便于利用的数据格式;针对数据缺失问题,需要采用缺失处相邻部分的数据进行填充,考虑到填充过程不应当引入噪声,因此采用matlab工具箱中的移动均值方法填补缺失数据;对于数据个体多样问题,由于每个个体的动作分析应当独立进行,且需要分析的个体类型为“Air+FixedWing”且每次战斗不止有一架飞机参与战斗,因此需要使用id和类型对时间序列进行分类分析。

在分析数据的基础上,我们还构思了提取飞行参数变化情况的方法。单一飞机一段时间内的飞行数据,可以看作是多维度的时间序列数据。为了分析各个维度的变化状态,尝试使用差分法,但是体现出易受噪声干扰和局部性的特点,不能很好体现出数据的变化趋势。由于以上原因,又改用双滑动窗口法,具有较好的消除噪声和判断趋势的能力。

\subsection{问题三的分析}


%----------- 三、模型假设 ----------
\section{三、模型假设}
%使用代码片段:、jiashe%
\begin{enumerate}
    \item 飞机所记录的飞行参数真实可信,没有因各种因素而导致数据错误。
    
    \textbf{原因:}动作序列的分割识别工作基于飞行时的各项参数所决定,如果飞行参数有误,那么所计算得出的动作序列将不准确。

    \item 十大
    
    \textbf{原因:}
    
\end{enumerate}

%----------- 四、符号说明 ----------
\section{四、名词解释与符号说明}
%使用三线表格最好~
\subsection{名词解释}
\begin{enumerate}
    % 名词:、mingci
    \item \textbf{飞行参数}
    
    在空战模拟中飞机的时序数据,包含飞机海拔高度,真实空速(TAS),俯仰角,偏航角(Yaw)等指标。每条指标由id与Unix时间唯一标识。
    
    \item \textbf{机动动作}
    
    战机在空中飞行过程中,飞机为了某些战术意图而做出的行为。在分析过程中,机动动作是最小的分析单位,不能再进行分割。

    
\end{enumerate}
\subsection{符号说明}
以下是本文使用的符号以及含义:
\begin{table}[h]%htbp表示的意思是latex会尽量满足排在前面的浮动格式,就是h-t-b-p这个顺序,让排版的效果尽量好。
    \centering
    \begin{tabular}{p{2.0cm}<{\centering}p{9.0cm}<{\centering}p{2.0cm}<{\centering}}
 %指定单元格宽度, 并且水平居中。
    \hline
    符号 & 说明 & 单位 \\ %换行 
    \hline
    $L_0$ & 仓库长度 &  $m$\\
    
    \hline
    \end{tabular}
\end{table}

%----------- 五、模型的建立与求解 ----------
\section{五、模型的建立与求解}

以下将对提出的三个问题进行建模求解。

\subsection{机动动作描述与量化模型}

为了进行空战势态感知,机动决策,意图识别等工作,常常根据需要预先建立空战动作库\cite{1}。常见的空战库设计有两种方法,分别是包含丰富战术动作的典型战术动作库和由美国NASA学者\cite{2}提出的基本操纵动作库。前者内容丰富,但是前一种方法存在识别困难,对于某些复杂动作可能中断的情况处理不佳,而后一种方式以极限情况操作粗猛,不能保证组合出所有的战术动作。文献\cite{3}中提出的机动动作集合兼顾二者的优点,列举了仪表动作,简单特技,复杂特技共三类十二种动作,如表(\ref{dongzuo}) 。这一选择兼顾了简单动作与复杂动作,具有飞行动作代表性,易于识别。

\begin{table}[h]
\centering
\vspace{3pt}
\centering
  \setlength{\leftskip}{0pt plus 1fil minus \marginparwidth}%通过这三行来改变三线表的对齐方式和宽度
  \setlength{\rightskip}{\leftskip}
  \resizebox{1.2\textwidth}{!}{
\begin{tabular}{>{ \centering \arraybackslash}p{5em}>{ \centering \arraybackslash}p{30em}}
\toprule % 绘制第一条线
   A & b\\
\midrule
 t & DW  \\ 
\bottomrule
\end{tabular}}
\caption{{caption:default}}
\end{table}
\section{七、模型的评价}

\subsection{模型的优点}
\begin{enumerate}
    \item 采用

\end{enumerate}

\subsection{模型的缺点}
\begin{itemize}
    \item 利用较

\end{itemize}

%----------- 参考文献 ----------
\newpage
\bibliographystyle{unsrt} %规定了参考文献的格式
\begin{center}
\bibliography{reference} %调出LaTeX生成参考文献列表
\end{center}

%----------- 附录 ----------
\newpage
\section{附件}
\textbf{附件清单:}
\begin{itemize}
    \item xxx代码
\end{itemize}

\textbf{sobel边缘检测代码}

\begin{lstlisting}[language=matlab]
    function GAdsa 
\end{lstlisting}



\end{document}