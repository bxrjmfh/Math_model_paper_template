\documentclass{my_paper}
\usepackage{ctex}
\usepackage[textwidth=444bp,vmargin=2.5cm]{geometry}%设置页边距
\usepackage{array} %主要是增加列样式选项
\usepackage[dvipsnames]{xcolor}%颜色宏包
\usepackage{graphicx}%图片宏包
\usepackage{amsmath}%公式宏包
\usepackage[T1]{fontenc}    
\usepackage{newtxtext, newtxmath}  %两种使用Times New Roman 字体的方法
\usepackage{subfigure}
\usepackage{tabularx, booktabs} %% Load packages that you use
\usepackage{multirow} %跨行处理
\usepackage{rotating}%横向表格
\usepackage{diagbox}%斜线划分表头

\usepackage { gensymb }
% 打°符号\degree
\usepackage{framed}
\usepackage{listings}
% 代码
\usepackage{color} %red, green, blue, yellow, cyan, magenta, black, white
\usepackage[numbered,framed]{matlab-prettifier}%matlab 代码高亮
\usepackage{mdframed}%另一个边框
% matlab代码样式,使用方法为:
% \lstinputlisting[style=Matlab-editor,linewidth=\textwidth]{code.m}
% 或:
% \begin{lstlisting}[style=matlab-prettifier]
%     %code
% \end{lstlisting}
\renewenvironment{framed}[1][\hsize]
  {\MakeFramed{\hsize#1\advance\hsize-\width \FrameRestore}}%
  {\endMakeFramed}
%   修正framed环境,使之可以变长,用法:
%   \begin{framed}[1.2/textwidth]...

\usepackage{hologo}
\usepackage{gbt7714}
\bibliographystyle{gbt7714-numerical}
% 采用国标参考文献引用
\newcommand{\lunwenbiaoti}{\fontsize{15.75pt}{0}\heiti 基于PID自动控制的暖气房屋控温模型}
\newcommand{\zhaiyao}{\fontsize{14pt}{0}\heiti 摘要}
    
\begin{document}

\lstdefinestyle{python_style}{
 columns=fixed,
 numbers=left,                                        % 在左侧显示行号
 numberstyle=\tiny\color{gray},                       % 设定行号格式
 frame=trbl,                                        % 单线背景边框
 breaklines=true,                                     % 设定LaTeX对过长的代码行进行自动换行
 keywordstyle=\color[RGB]{40,40,255},                 % 设定关键字颜色
 numberstyle=\footnotesize\color{darkgray},
 commentstyle=\it\color[RGB]{0,96,96},                % 设置代码注释的格式
 stringstyle=\rmfamily\slshape\color[RGB]{128,0,0},   % 设置字符串格式
 showstringspaces=false,                              % 不显示字符串中的空格
 language=python,                                        % 设置语言
 basicstyle=\linespread{1.0}\fontsize{10bp}{10bp}\selectfont\ttfamily,                      % 字体字号
 %lineskip=10bp,
 %baselinestretch=1,
}
\newpage
\begin{center}
\lunwenbiaoti

\vspace{2ex}
\zhaiyao
\end{center}

摘要

\begin{guanjianci}
 元胞自动机 \quad 边缘检测 \quad 形状匹配
\end{guanjianci}

%----------- 正文 ----------
%----------- 一、问题重述 ----------
\newpage
\section{一、问题重述}

\subsection{问题背景}

一些背景

\subsection{问题重述}
经过分析整理,我们需要解决以下问题:
\begin{enumerate}
    \item 考虑到
    \item 在自卸
    \item 在第二
\end{enumerate}
\section{二、问题分析}
\subsection{问题一的分析}

分析

\subsection{问题二的分析}

分析

\subsection{问题三的分析}


%----------- 三、模型假设 ----------
\section{三、模型假设}
%使用代码片段:、jiashe%
\begin{enumerate}
    \item 
    
    \textbf{原因:}


\end{enumerate}

%----------- 四、符号说明 ----------
\section{四、名词解释与符号说明}
%使用三线表格最好~
\subsection{名词解释}
\begin{enumerate}
    % 名词:、mingci
    \item \textbf{dada}
    
    dsadw
    
    \item \textbf{dsadc}
    
    dasdsas

    
\end{enumerate}
\subsection{符号说明}
以下是本文使用的符号以及含义:
\begin{table}[h]%htbp表示的意思是latex会尽量满足排在前面的浮动格式,就是h-t-b-p这个顺序,让排版的效果尽量好。
    \centering
    \begin{tabular}{p{2.0cm}<{\centering}p{9.0cm}<{\centering}p{2.0cm}<{\centering}}
 %指定单元格宽度, 并且水平居中。
    \hline
    符号 & 说明 & 单位 \\ %换行 
    \hline
    $L_0$ & 仓库长度 &  $m$\\
    
    \hline
    \end{tabular}
\end{table}

%----------- 五、模型的建立与求解 ----------
\section{五、模型的建立与求解}

以下将对提出的三个问题进行建模求解。
\section{六、敏感性分析}
\section{七、模型的评价}

\subsection{模型的优点}
\begin{enumerate}
    \item 采用

\end{enumerate}

\subsection{模型的缺点}
\begin{itemize}
    \item 利用较

\end{itemize}

%----------- 参考文献 ----------
\newpage
\begin{center}
\bibliography{reference} %调出LaTeX生成参考文献列表
\end{center}

%----------- 附录 ----------
\newpage
\section{附件}
\textbf{附件清单:}
\renewcommand\theenumi{\roman{enumi}}
% 规定数字格式为罗马数字
\renewcommand\labelenumi{\textbf{附录\theenumi}}
% 规定是附录某某
\begin{itemize}
    % \lstinputlisting[style=Matlab-editor,linewidth=\textwidth]{code.m}
\end{itemize}



% \lstinputlisting[style=Matlab-editor,linewidth=\textwidth]{code.m}




\end{document}